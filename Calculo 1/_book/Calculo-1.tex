\documentclass[]{book}
\usepackage{lmodern}
\usepackage{amssymb,amsmath}
\usepackage{ifxetex,ifluatex}
\usepackage{fixltx2e} % provides \textsubscript
\ifnum 0\ifxetex 1\fi\ifluatex 1\fi=0 % if pdftex
  \usepackage[T1]{fontenc}
  \usepackage[utf8]{inputenc}
\else % if luatex or xelatex
  \ifxetex
    \usepackage{mathspec}
  \else
    \usepackage{fontspec}
  \fi
  \defaultfontfeatures{Ligatures=TeX,Scale=MatchLowercase}
\fi
% use upquote if available, for straight quotes in verbatim environments
\IfFileExists{upquote.sty}{\usepackage{upquote}}{}
% use microtype if available
\IfFileExists{microtype.sty}{%
\usepackage{microtype}
\UseMicrotypeSet[protrusion]{basicmath} % disable protrusion for tt fonts
}{}
\usepackage[margin=1in]{geometry}
\usepackage{hyperref}
\hypersetup{unicode=true,
            pdftitle={BGuide - Cálculo 1},
            pdfauthor={Bruno Geronymo},
            pdfborder={0 0 0},
            breaklinks=true}
\urlstyle{same}  % don't use monospace font for urls
\usepackage{natbib}
\bibliographystyle{apalike}
\usepackage{longtable,booktabs}
\usepackage{graphicx,grffile}
\makeatletter
\def\maxwidth{\ifdim\Gin@nat@width>\linewidth\linewidth\else\Gin@nat@width\fi}
\def\maxheight{\ifdim\Gin@nat@height>\textheight\textheight\else\Gin@nat@height\fi}
\makeatother
% Scale images if necessary, so that they will not overflow the page
% margins by default, and it is still possible to overwrite the defaults
% using explicit options in \includegraphics[width, height, ...]{}
\setkeys{Gin}{width=\maxwidth,height=\maxheight,keepaspectratio}
\IfFileExists{parskip.sty}{%
\usepackage{parskip}
}{% else
\setlength{\parindent}{0pt}
\setlength{\parskip}{6pt plus 2pt minus 1pt}
}
\setlength{\emergencystretch}{3em}  % prevent overfull lines
\providecommand{\tightlist}{%
  \setlength{\itemsep}{0pt}\setlength{\parskip}{0pt}}
\setcounter{secnumdepth}{5}
% Redefines (sub)paragraphs to behave more like sections
\ifx\paragraph\undefined\else
\let\oldparagraph\paragraph
\renewcommand{\paragraph}[1]{\oldparagraph{#1}\mbox{}}
\fi
\ifx\subparagraph\undefined\else
\let\oldsubparagraph\subparagraph
\renewcommand{\subparagraph}[1]{\oldsubparagraph{#1}\mbox{}}
\fi

%%% Use protect on footnotes to avoid problems with footnotes in titles
\let\rmarkdownfootnote\footnote%
\def\footnote{\protect\rmarkdownfootnote}

%%% Change title format to be more compact
\usepackage{titling}

% Create subtitle command for use in maketitle
\newcommand{\subtitle}[1]{
  \posttitle{
    \begin{center}\large#1\end{center}
    }
}

\setlength{\droptitle}{-2em}
  \title{BGuide - Cálculo 1}
  \pretitle{\vspace{\droptitle}\centering\huge}
  \posttitle{\par}
  \author{Bruno Geronymo}
  \preauthor{\centering\large\emph}
  \postauthor{\par}
  \predate{\centering\large\emph}
  \postdate{\par}
  \date{2018-03-04}

\usepackage{booktabs}
\usepackage{amsthm}
\makeatletter
\def\thm@space@setup{%
  \thm@preskip=8pt plus 2pt minus 4pt
  \thm@postskip=\thm@preskip
}
\makeatother

\begin{document}
\maketitle

{
\setcounter{tocdepth}{1}
\tableofcontents
}
\chapter*{Prefácio}\label{prefacio}
\addcontentsline{toc}{chapter}{Prefácio}

Este material trata-se de um manual de resoluções dos exercícios
propostos no livro \emph{Um Curso de Cálculo, Volume 1} de
\emph{Hamilton Luiz Guidorizzi}. Ao decorrer das resoluções o material
busca apresentar, adicionalmente, resoluções computacionais através do
software \href{https://www.r-project.org/}{R de computação estatística}
para facilitar a visualização do problema e também o aprendizado da
linguagem \emph{R}.

O material procura abordar todos os assuntos tratados no livro do
\emph{Guidorizzi}, seguindo também a mesma ordem dos capítulos, para
facilitar a dinâmica de pesquisa por assuntos específicos.

\chapter{Números reais}\label{numeros-reais}

\section{Os Números Racionais}\label{os-numeros-racionais}

Por uma questão de notação admitiremos aqui que, sendo \(r\) um número
racional, se \(r \leqslant 0\), dizemos que \(r\) é não positivo. Da
mesma forma, se \(r \geqslant 0\), dizemos que \(r\) é não negativo.

Vale acrescentar aqui algumas definições que poderão auxiliar na leitura
do livro.

\begin{itemize}
\tightlist
\item
  \textbf{Abscissa}: Trata-se da coordenada de um ponto sobre uma reta.
\item
  \textbf{Irredutível}: Algo que não se pode reduzir. Uma fração é dita
  irredutível quando está em sua forma mais reduzida possível.
\end{itemize}

\chapter{Sem Título}\label{sem-titulo}

\chapter{Sem Título}\label{sem-titulo-1}

\chapter{Sem Título}\label{sem-titulo-2}

\chapter{Sem Título}\label{sem-titulo-3}

\bibliography{book.bib,packages.bib}


\end{document}
