\documentclass[]{book}
\usepackage{lmodern}
\usepackage{amssymb,amsmath}
\usepackage{ifxetex,ifluatex}
\usepackage{fixltx2e} % provides \textsubscript
\ifnum 0\ifxetex 1\fi\ifluatex 1\fi=0 % if pdftex
  \usepackage[T1]{fontenc}
  \usepackage[utf8]{inputenc}
\else % if luatex or xelatex
  \ifxetex
    \usepackage{mathspec}
  \else
    \usepackage{fontspec}
  \fi
  \defaultfontfeatures{Ligatures=TeX,Scale=MatchLowercase}
\fi
% use upquote if available, for straight quotes in verbatim environments
\IfFileExists{upquote.sty}{\usepackage{upquote}}{}
% use microtype if available
\IfFileExists{microtype.sty}{%
\usepackage{microtype}
\UseMicrotypeSet[protrusion]{basicmath} % disable protrusion for tt fonts
}{}
\usepackage[margin=1in]{geometry}
\usepackage{hyperref}
\hypersetup{unicode=true,
            pdftitle={BGuide - Cálculo 1},
            pdfauthor={Bruno Geronymo},
            pdfborder={0 0 0},
            breaklinks=true}
\urlstyle{same}  % don't use monospace font for urls
\usepackage{natbib}
\bibliographystyle{apalike}
\usepackage{color}
\usepackage{fancyvrb}
\newcommand{\VerbBar}{|}
\newcommand{\VERB}{\Verb[commandchars=\\\{\}]}
\DefineVerbatimEnvironment{Highlighting}{Verbatim}{commandchars=\\\{\}}
% Add ',fontsize=\small' for more characters per line
\usepackage{framed}
\definecolor{shadecolor}{RGB}{248,248,248}
\newenvironment{Shaded}{\begin{snugshade}}{\end{snugshade}}
\newcommand{\KeywordTok}[1]{\textcolor[rgb]{0.13,0.29,0.53}{\textbf{#1}}}
\newcommand{\DataTypeTok}[1]{\textcolor[rgb]{0.13,0.29,0.53}{#1}}
\newcommand{\DecValTok}[1]{\textcolor[rgb]{0.00,0.00,0.81}{#1}}
\newcommand{\BaseNTok}[1]{\textcolor[rgb]{0.00,0.00,0.81}{#1}}
\newcommand{\FloatTok}[1]{\textcolor[rgb]{0.00,0.00,0.81}{#1}}
\newcommand{\ConstantTok}[1]{\textcolor[rgb]{0.00,0.00,0.00}{#1}}
\newcommand{\CharTok}[1]{\textcolor[rgb]{0.31,0.60,0.02}{#1}}
\newcommand{\SpecialCharTok}[1]{\textcolor[rgb]{0.00,0.00,0.00}{#1}}
\newcommand{\StringTok}[1]{\textcolor[rgb]{0.31,0.60,0.02}{#1}}
\newcommand{\VerbatimStringTok}[1]{\textcolor[rgb]{0.31,0.60,0.02}{#1}}
\newcommand{\SpecialStringTok}[1]{\textcolor[rgb]{0.31,0.60,0.02}{#1}}
\newcommand{\ImportTok}[1]{#1}
\newcommand{\CommentTok}[1]{\textcolor[rgb]{0.56,0.35,0.01}{\textit{#1}}}
\newcommand{\DocumentationTok}[1]{\textcolor[rgb]{0.56,0.35,0.01}{\textbf{\textit{#1}}}}
\newcommand{\AnnotationTok}[1]{\textcolor[rgb]{0.56,0.35,0.01}{\textbf{\textit{#1}}}}
\newcommand{\CommentVarTok}[1]{\textcolor[rgb]{0.56,0.35,0.01}{\textbf{\textit{#1}}}}
\newcommand{\OtherTok}[1]{\textcolor[rgb]{0.56,0.35,0.01}{#1}}
\newcommand{\FunctionTok}[1]{\textcolor[rgb]{0.00,0.00,0.00}{#1}}
\newcommand{\VariableTok}[1]{\textcolor[rgb]{0.00,0.00,0.00}{#1}}
\newcommand{\ControlFlowTok}[1]{\textcolor[rgb]{0.13,0.29,0.53}{\textbf{#1}}}
\newcommand{\OperatorTok}[1]{\textcolor[rgb]{0.81,0.36,0.00}{\textbf{#1}}}
\newcommand{\BuiltInTok}[1]{#1}
\newcommand{\ExtensionTok}[1]{#1}
\newcommand{\PreprocessorTok}[1]{\textcolor[rgb]{0.56,0.35,0.01}{\textit{#1}}}
\newcommand{\AttributeTok}[1]{\textcolor[rgb]{0.77,0.63,0.00}{#1}}
\newcommand{\RegionMarkerTok}[1]{#1}
\newcommand{\InformationTok}[1]{\textcolor[rgb]{0.56,0.35,0.01}{\textbf{\textit{#1}}}}
\newcommand{\WarningTok}[1]{\textcolor[rgb]{0.56,0.35,0.01}{\textbf{\textit{#1}}}}
\newcommand{\AlertTok}[1]{\textcolor[rgb]{0.94,0.16,0.16}{#1}}
\newcommand{\ErrorTok}[1]{\textcolor[rgb]{0.64,0.00,0.00}{\textbf{#1}}}
\newcommand{\NormalTok}[1]{#1}
\usepackage{longtable,booktabs}
\usepackage{graphicx,grffile}
\makeatletter
\def\maxwidth{\ifdim\Gin@nat@width>\linewidth\linewidth\else\Gin@nat@width\fi}
\def\maxheight{\ifdim\Gin@nat@height>\textheight\textheight\else\Gin@nat@height\fi}
\makeatother
% Scale images if necessary, so that they will not overflow the page
% margins by default, and it is still possible to overwrite the defaults
% using explicit options in \includegraphics[width, height, ...]{}
\setkeys{Gin}{width=\maxwidth,height=\maxheight,keepaspectratio}
\IfFileExists{parskip.sty}{%
\usepackage{parskip}
}{% else
\setlength{\parindent}{0pt}
\setlength{\parskip}{6pt plus 2pt minus 1pt}
}
\setlength{\emergencystretch}{3em}  % prevent overfull lines
\providecommand{\tightlist}{%
  \setlength{\itemsep}{0pt}\setlength{\parskip}{0pt}}
\setcounter{secnumdepth}{5}
% Redefines (sub)paragraphs to behave more like sections
\ifx\paragraph\undefined\else
\let\oldparagraph\paragraph
\renewcommand{\paragraph}[1]{\oldparagraph{#1}\mbox{}}
\fi
\ifx\subparagraph\undefined\else
\let\oldsubparagraph\subparagraph
\renewcommand{\subparagraph}[1]{\oldsubparagraph{#1}\mbox{}}
\fi

%%% Use protect on footnotes to avoid problems with footnotes in titles
\let\rmarkdownfootnote\footnote%
\def\footnote{\protect\rmarkdownfootnote}

%%% Change title format to be more compact
\usepackage{titling}

% Create subtitle command for use in maketitle
\newcommand{\subtitle}[1]{
  \posttitle{
    \begin{center}\large#1\end{center}
    }
}

\setlength{\droptitle}{-2em}
  \title{BGuide - Cálculo 1}
  \pretitle{\vspace{\droptitle}\centering\huge}
  \posttitle{\par}
  \author{Bruno Geronymo}
  \preauthor{\centering\large\emph}
  \postauthor{\par}
  \predate{\centering\large\emph}
  \postdate{\par}
  \date{2018-03-04}

\usepackage{booktabs}
\usepackage{amsthm}
\makeatletter
\def\thm@space@setup{%
  \thm@preskip=8pt plus 2pt minus 4pt
  \thm@postskip=\thm@preskip
}
\makeatother

\begin{document}
\maketitle

{
\setcounter{tocdepth}{1}
\tableofcontents
}
\chapter*{Prefácio}\label{prefacio}
\addcontentsline{toc}{chapter}{Prefácio}

Este material trata-se de um manual de resoluções dos exercícios
propostos no livro \emph{Um Curso de Cálculo, Volume 1, 5ª Edição} de
\emph{Hamilton Luiz Guidorizzi}. Ao decorrer das resoluções o material
busca apresentar, adicionalmente, resoluções computacionais através do
software \href{https://www.r-project.org/}{R de computação estatística}
para facilitar a visualização do problema e também o aprendizado da
linguagem \emph{R}.

O material procura abordar todos os assuntos tratados no livro do
\emph{Guidorizzi}, seguindo também a mesma ordem dos capítulos, para
facilitar a dinâmica de pesquisa por assuntos específicos.

\chapter{Números reais}\label{numeros-reais}

\section{Os Números Racionais}\label{os-numeros-racionais}

Por uma questão de notação admitiremos aqui que, sendo \(r\) um número
racional, se \(r \leqslant 0\), dizemos que \(r\) é não positivo. Da
mesma forma, se \(r \geqslant 0\), dizemos que \(r\) é não negativo.

Vale acrescentar aqui algumas definições que poderão auxiliar na leitura
do livro.

\begin{itemize}
\tightlist
\item
  \textbf{Abscissa}: Trata-se da coordenada de um ponto sobre uma reta.
\item
  \textbf{Irredutível}: Algo que não se pode reduzir. Uma fração é dita
  irredutível quando está em sua forma mais reduzida possível.
\end{itemize}

\section{Os Números Reais}\label{os-numeros-reais}

\textbf{EXEMPLO 4.} \emph{(Página 6)} Suponha \(x \geqslant 0\) e
\(y \geqslant 0\). Prove:

\begin{enumerate}
\def\labelenumi{\alph{enumi})}
\setcounter{enumi}{1}
\tightlist
\item
  \(x \leqslant y \Rightarrow x^{2} \leqslant y^{2}\).
\end{enumerate}

~~~~~~\emph{Resolução}:

\[\textrm{e} \ \left.\begin{matrix} 0 \leqslant x \leqslant y\\ 0 \leqslant x \leqslant y \end{matrix}\right\} \overset{(OM)}{\Rightarrow} \ \textrm{e} \ \left.\begin{matrix} xx \leqslant xy \\ xy \leqslant yy \end{matrix}\right\} \overset{(O3)}{\Rightarrow} xx \leqslant xy \leqslant yy \Rightarrow xx \leqslant yy \Rightarrow x^{2} \leqslant y^{2}\]

\begin{Shaded}
\begin{Highlighting}[]
\CommentTok{# Estudo por simulação:}

\NormalTok{## Semente:}

\KeywordTok{set.seed}\NormalTok{(}\KeywordTok{sum}\NormalTok{(}\KeywordTok{utf8ToInt}\NormalTok{(}\StringTok{"BGuide"}\NormalTok{)))}

\NormalTok{## Quantidade de números a serem gerados:}

\NormalTok{n <-}\StringTok{ }\DecValTok{1000000}

\NormalTok{##  Gera-se aqui 'n' números aleatórios seguindo a distribuição Uniforme de}
\NormalTok{## parâmetros 'min = 0' e 'max = 1':}

\NormalTok{x <-}\StringTok{ }\KeywordTok{runif}\NormalTok{(n)}

\NormalTok{##  Em seguida geramos mais 'n' números aleatórios seguindo uma distribuição}
\NormalTok{## Uniforme de parâmetros 'min = x' e 'max = 1'. Isto faz com que todos os}
\NormalTok{## números armazenados em y[i] sejam maiores do que os armazenados em x[i], com}
\NormalTok{## 'i' variando de 1 a 'n'. Mas não implica que y[i] seja maior do que x[j] com}
\NormalTok{## 'j' também variando de 1 a 'n' e 'i != j':}

\NormalTok{y <-}\StringTok{ }\KeywordTok{runif}\NormalTok{(n, }\DataTypeTok{min =}\NormalTok{ x)}

\NormalTok{##  Soma a quantidade de verificações onde a afirmação 'x^2 <= y^2' for}
\NormalTok{## verdadeira:}

\KeywordTok{sum}\NormalTok{(x}\OperatorTok{^}\DecValTok{2} \OperatorTok{<=}\StringTok{ }\NormalTok{y}\OperatorTok{^}\DecValTok{2}\NormalTok{)}
\end{Highlighting}
\end{Shaded}

\begin{verbatim}
## [1] 1000000
\end{verbatim}

\begin{Shaded}
\begin{Highlighting}[]
\NormalTok{##  Observe que o resultado é 1.000.000, exatamente a quantidade de números}
\NormalTok{## uniformes no intervalo (0, 1) que foram gerados. Logo para todas as}
\NormalTok{## simulações obteve-se 'x^2 <= y^2'.}

\NormalTok{##  Obs.: O resultado obtido por simulação não prova a propriedade acima, apenas}
\NormalTok{## cria evidências a favor dela. A simulação não é necessária aqui pois a}
\NormalTok{## propriedade pode ser provada analiticamente.}
\end{Highlighting}
\end{Shaded}

~

\textbf{EXEMPLO 9.} \emph{(Página 10)} Resolva a inequação
\(\frac{3x-1}{x+2} \geqslant 5\).

Sendo \(x < 2\):

\[\frac{3x-1}{x+2} \geqslant 5 \Leftrightarrow 3x-1 \leqslant 5(x+2).\]

Então o autor pergunta: Por quê?

Sabemos que \(1 < 2\), se multiplicássemos esta expressão por \(-1\) sem
alterarmos o sentido da desigualdade teríamos \(-1 < -2\) e sabemos que
esta afirmação não é verdadeira. Considerando \(a < 0\), se
multiplicarmos uma desigualdade por \(a\) altera-se a sentido da
desigualdade pois refletimos estes valores para o outro lado de um eixo
com relação a origem a uma taxa de progressão \(\left | a \right |\).
Porém, ao realizar este processo a direção de crescimento das unidades
permanece a mesma (não é refletida).

\begin{Shaded}
\begin{Highlighting}[]
\CommentTok{# Exemplo gráfico}

\NormalTok{##  O gráfico a seguir tem como objetivo a visualização do que foi dito}
\NormalTok{## anteriormente. Nele pode-se observar a propriedade 'x <= y  ==>  ky <= kx'}
\NormalTok{## quando 'k < 0'.}

\NormalTok{## Cria um vetor dos valores a serem plotados:}
\NormalTok{xy <-}\StringTok{ }\KeywordTok{c}\NormalTok{(}\DecValTok{2}\NormalTok{, }\DecValTok{3}\NormalTok{)}
\NormalTok{k <-}\StringTok{ }\OperatorTok{-}\DecValTok{2}
\NormalTok{pontos <-}\StringTok{ }\KeywordTok{c}\NormalTok{(k}\OperatorTok{*}\NormalTok{xy, xy)}

\NormalTok{## Cria gráfico unidimensional:}
\KeywordTok{stripchart}\NormalTok{(pontos, }\DataTypeTok{col =} \StringTok{"red"}\NormalTok{, }\DataTypeTok{lwd =} \DecValTok{3}\NormalTok{, }\DataTypeTok{xlim =} \KeywordTok{c}\NormalTok{(}\OperatorTok{-}\DecValTok{8}\NormalTok{, }\DecValTok{4}\NormalTok{))}

\NormalTok{## Linha da origem:}
\KeywordTok{abline}\NormalTok{(}\DataTypeTok{v =} \DecValTok{0}\NormalTok{, }\DataTypeTok{col =} \StringTok{"red"}\NormalTok{, }\DataTypeTok{lwd =} \DecValTok{2}\NormalTok{)}

\NormalTok{## Eixo do sistema:}
\KeywordTok{arrows}\NormalTok{(}\DataTypeTok{x0 =} \OperatorTok{-}\DecValTok{8}\NormalTok{, }\DataTypeTok{y0 =} \DecValTok{1}\NormalTok{, }\DataTypeTok{x1 =} \DecValTok{4}\NormalTok{, }\DataTypeTok{y1 =} \DecValTok{1}\NormalTok{, }\DataTypeTok{lwd =} \DecValTok{2}\NormalTok{)}

\NormalTok{## Coordenadas da legenda:}
\NormalTok{x <-}\StringTok{ }\KeywordTok{c}\NormalTok{(}\OperatorTok{-}\FloatTok{6.007753}\NormalTok{, }\OperatorTok{-}\FloatTok{4.010723}\NormalTok{, }\FloatTok{1.992396}\NormalTok{, }\FloatTok{2.990910}\NormalTok{)}
\NormalTok{y <-}\StringTok{ }\KeywordTok{rep}\NormalTok{(}\FloatTok{1.05}\NormalTok{, }\DecValTok{4}\NormalTok{)}

\NormalTok{## Legenda:}
\KeywordTok{text}\NormalTok{(x, y, }\DataTypeTok{labels =} \KeywordTok{c}\NormalTok{(}\StringTok{"ky"}\NormalTok{, }\StringTok{"kx"}\NormalTok{, }\StringTok{"x"}\NormalTok{, }\StringTok{"y"}\NormalTok{), }\DataTypeTok{col =} \StringTok{"red"}\NormalTok{, }\DataTypeTok{lwd =} \DecValTok{3}\NormalTok{)}
\end{Highlighting}
\end{Shaded}

\includegraphics{Calculo-1_files/figure-latex/unnamed-chunk-2-1.pdf}

\chapter{Sem Título}\label{sem-titulo}

\chapter{Sem Título}\label{sem-titulo-1}

\chapter{Sem Título}\label{sem-titulo-2}

\chapter{Sem Título}\label{sem-titulo-3}

\bibliography{book.bib,packages.bib}


\end{document}
